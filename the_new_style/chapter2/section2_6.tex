\documentclass[11pt]{amsart}
\usepackage[margin=1in]{geometry}
\usepackage{paralist}
\usepackage{graphicx}
\usepackage{mathtools}

\theoremstyle{definition}
\newtheorem{exercise}{Exercise}
\newtheorem*{noexercise}{Exercise}
\newtheorem*{theorem}{Theorem}

\newenvironment{amatrix}[1]{%
  \left(\begin{array}{@{}*{#1}{c}|c@{}}
}{%
  \end{array}\right)
}

\begin{document}
\title{Linear Algebra}
\author{Strang, Section 2.6}
\maketitle

\section{The assignment}
\begin{compactitem}
\item Read section 2.6 of Strang (pages 95 - 102).
\item (Optional) Read the topics \emph{Computer Algebra Systems} and \emph{Accuracy of Computations} at the end of Chapter One of Hefferon (pages 65 - 71).
\item Read the following and complete the exercises below.
\end{compactitem}


\section{The $LU$ decomposition of a matrix}

We now look at the ideas behind elimination from a more advanced perspective. If we think about the matrix multiplication form of the forward pass, we can realize it a \emph{matrix decomposition theorem}:

\begin{theorem} Any square matrix $A$ can be written as a product $A = LU$ where $L$ is a lower triangular matrix and $U$ is an upper triangular matrix. Moreover, the matrix $L$ will have $1$'s down its diagonal.
\end{theorem}

There are three key observations that make this work:
\begin{itemize}
\item Each of the matrices $E_{ij}$ that affects a row operation of the form \emph{add a multiple of row $i$ to row $j$} is an invertible matrix, with an easy to find inverse.

\item If we make a sequence of row operations in the forward pass using matrices $E_k$, then we are essentially computing a big product
\[
E_k \dots E_1 A = U
\]
where each of the $E_i$'s is a lower triangular matrix and the matrix $U$ is upper triangular. This can the be rewritten as
\[
A = \left(E_1^{-1} \dots E_k^{-1} \right) U .
\]
Note that the inverses have to be done \emph{in reverse order} for things to cancel out properly.

\item Finally, the product $L = E_1^{-1} \dots E_k^{-1}$ is really easy to compute, because its entries are simply the negatives of the multipliers we used to do the operations in the forward pass.
\end{itemize}

\subsection{A Nice Computational Result}

One important output of this comes into play when we want to compute solutions to equations like $Ax = b$. Since we can write $A = LU$, then our equation can be split into two (big) steps:
\begin{enumerate}
\item First find the solution to the equation $Ly = b$.

\item Then find the solution to the equation $Ux = y$.
\end{enumerate}

First, note that this is a good thing because both of the systems $Ly = b$ and $Ux = y$ are triangular. They can be solved by back substitution. In $Ly = b$ you work from the top down, and in $Ux=y$ you work from the bottom up.

Second, this works because following this process gives us a vector $x$ which will satisfy this:
\[
Ax = (LU)x = L (Ux) = Ly = b.
\]

Third, this doesn't really save time when you only want to solve one equation $Ax= b$. But if you have lots of different values of $b_i$, and you want to solve all of the equations $Ax = b_i$, it becomes a lot faster to factor the matrix $A= LU$ once and do two back substitutions for each value of $b_i$.



\section{Sage instructions}

I have made a Sage worksheet file with some basic commands that you might find useful when dealing with the $LU$ decomposition. The file is called \texttt{section2\_6.sagews}.


\section{Questions for Section 2.6}
\setcounter{exercise}{64}

Keep this in mind. The computations are simple, but tedious. Perhaps you want to use an appropriate tool.

\begin{exercise}
Consider the following system of 3 linear equations in 3 unknowns.
\[\left\{
\begin{array}{rrrrrrr}
x & + & y & + & z & = & 5 \\
x & + & 2y & + & 3z & = & 7 \\
x & + & 3y & + & 6z & = & 11
\end{array}\right.
\]
Perform the forward pass of elimination to find an equivalent upper triangular system. Write down this upper triangular system. What three row operations do you need to perform to make this work?

Use the information you just found to write a matrix decomposition $A = LU$ for the coefficient matrix $A$ for this system of equations. (Be sure to multiply the matrices $L$ and $U$ to check your work.)

\end{exercise}

\begin{exercise}
Solve the two systems $Ly = b$ and $Ux=y$ obtained in the last exercise.

Solve the system $Ax=b$ directly using Gauss-Jordan elimination (hint: use Sage) and make sure that the results are the same.
\end{exercise}

\begin{exercise}
Consider the matrix $A$ below. Find the matrix $E$ which transforms $A$ into an upper triangular matrix $EA = U$. Find $L = E^{-1}$. Use this to write down the $LU$ decomposition $A= LU$ of $A$.
\[
A =
\begin{pmatrix}
2 & 1 & 0 \\
0 & 4 & 2 \\
6 & 3 & 5
\end{pmatrix}
\]
\end{exercise}

\begin{exercise}
The matrix below is \emph{symmetric}, because if you flip it across its main diagonal you get the same thing. Find the $LDU$ triple decomposition of this symmetric matrix.
\[
B =
\begin{pmatrix}
2 & 4 \\
4 & 11
\end{pmatrix}
\]
\end{exercise}

\begin{exercise}
The matrix below is \emph{symmetric}, because if you flip it across its main diagonal you get the same thing. Find the $LDU$ triple decomposition of this symmetric matrix.
\[
C =
\begin{pmatrix}
1 & 4 & 0 \\
4 & 12 & 4 \\
0 & 4 & 0
\end{pmatrix}
\]

\end{exercise}

\begin{exercise}
The matrix below is \emph{symmetric}, because if you flip it across its main diagonal you get the same thing. Find the $LU$ decomposition of this symmetric matrix.
\[
D =
\begin{pmatrix}
a & a & a & a \\
a & b & b & b \\
a & b & c & c \\
a & b & c & d
\end{pmatrix}
\]
What conditions on the variables $a$, $b$, $c$, and $d$ will guarantee that this matrix has four pivots.
\end{exercise}

\begin{exercise}
Find an example of a $3\times 3$ matrix $A$ which has all of its entries non-zero, so that the $LU$ decomposition has $U = I$, where $I$ is the identity matrix, or explain why no such example exists.
\end{exercise}

\end{document}




%sagemathcloud={"zoom_width":100}