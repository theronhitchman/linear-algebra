\documentclass[11pt]{amsart}
\usepackage[margin=1in]{geometry}
\usepackage{paralist}
\usepackage{graphicx}
\usepackage{mathtools}

\theoremstyle{definition}
\newtheorem{exercise}{Exercise}

\newenvironment{amatrix}[1]{%
  \left(\begin{array}{@{}*{#1}{c}|c@{}}
}{%
  \end{array}\right)
}

\begin{document}
\title{Linear Algebra}
\author{Strang, Section 2.2}
\maketitle

\section{The assignment}
\begin{compactitem}
\item Read section 2.2 of Strang (pages 45-51).
\item (Optional) Read Chapter One Part I.1 of Hefferon (pages 1-8).
\item Read the following and complete the exercises below.
\end{compactitem}


\section{Elimination for Solving Systems of Linear Equations}

Now we begin the process of learning how to solve a system of linear equations systematically through a process called \emph{elimination}.

\subsection{Some terminology}

A typical system looks something like this:
\[
\left\{
\begin{array}{rrrrrrrr}
3 x_1 & +&  2 x_2 & - & \pi x_3 & = & 0 \\
-4 x_1 & -& 33 x_2 & + & x_3 & = & 12
\end{array}\right.
\]

This situation is \emph{two} equations in \emph{three} unknowns. The unknowns here are the three numbers $x_1$, $x_2$ and $x_3$ for which we search. Usually, we bundle the numbers together as a vector $(x_1, x_2, x_3)$. If we can find a vector which makes all of the equations true simultaneously, we call that vector a \emph{solution}.

Keep in mind that the process involves eliminating instances of the variable below \emph{pivots}. Strang describes the process pretty well, and gives good examples. What Strang describes in this section is sometimes called \emph{the forward pass} elimination.

Watch out for situations which are \emph{singular} in that they have fewer pivots than unknowns. A system is called \emph{non-singular} if it has as many pivots as unknowns.

\subsection{Keeping track of things}

Playing with all of the equations is nice, but all that really matters is the collection of coefficients, and the numbers on the right hand sides of the equal signs.
Experienced solvers get tired of copying notation from line to line in a computation, so they only keep track of the matrix of coefficients, \emph{augmented} by the vector on the right-hand side. In the example above, that augmented matrix is
\[
\begin{amatrix}{3}
3 & 2 & -\pi & 0 \\ -4 & -33 & 1 & 12
\end{amatrix}
\]
All of the row operations can be performed on just this augmented matrix, without losing any of the essential information.


\section{Sage instructions}

I have made a Sage worksheet file with some basic commands that you might find useful in investigating the process of elimination. The file is called \texttt{section2\_2.sagews}.


\section{Questions for Section 2.2}
\setcounter{exercise}{35}

\begin{exercise}
Use the elimination method to transform this system into an easier one. (Can you make it triangular?) Circle the pivots in the final result.
\[
\left\{
\begin{array}{rrrrrrr}
2x & + & 3y & + &  z & = & 8\\
4x & + & 7y & + & 5z & = & 20 \\
   & - & 2y & + & 2z & = & 0
\end{array}
\right.
\]
What two operations do you use to do this efficiently?

Now use back substitution to solve the system.
\end{exercise}

\begin{exercise}
Because the last system can be transformed in two operations, there are three equivalent systems generated through the process (the original, the intermediate, and the final).

Make row picture plots for each of the three systems. [Hint: SAGE] How do the operations transform the pictures?
\end{exercise}


\begin{exercise}
Suppose that a system of three equations in three unknowns has two solutions $(x,y,z)$ and $(X,Y,Z)$. Explain why the system must have other solutions than these two. Describe clearly two other solutions in terms of $x,y,z,X,Y,Z$.
\end{exercise}


\begin{exercise}
For which three numbers $a$ will elimination fail to give three pivots for this coefficient matrix?
\[
A = \begin{pmatrix}
a & 2 & 3 \\ a & a & 4 \\ a & a & a
\end{pmatrix}
\]
\end{exercise}



\begin{exercise}
How many ways can two lines in the plane meet? Make examples to represent as many \emph{qualitatively different} situations as you can.
\end{exercise}


\begin{exercise}
How many ways can three planes in three dimensional space meet? Make examples to represent as many \emph{qualitatively different} situations as you can.
\end{exercise}


\begin{exercise}
Complete the following to make an example of a system of two equations in two unknowns which is singular and has a solution, or explain why no such example exists.
\[
\left\{
\begin{array}{ccccc}
2x & + & 3y & = & 1 \\
\bullet x & + & \bullet y & = & \bullet
\end{array}
\right.
\]
\end{exercise}


\begin{exercise}
Complete the following to a system of three equations in three unknowns which is singular and does not have a solution, or explain why no such example exists.
\[
\left\{
\begin{array}{ccccccc}
   &  & 3y & - & z & = & 1 \\
2x & - & y & + & 3z & = & 0 \\
\bullet x & + & \bullet y & + &\bullet z &  = & \bullet
\end{array}
\right.
\]
\end{exercise}


\begin{exercise}
Complete the following to a system of three equations in three unknowns which is singular and does have a solution, or explain why no such example exists.
\[
\left\{
\begin{array}{ccccccc}
 x & + & y & + & z & = & 1 \\
2x & + & y & + & 2z & = & 0 \\
\bullet x & + & \bullet y & + &\bullet z &  = & \bullet
\end{array}
\right.
\]

\end{exercise}


\end{document}




%sagemathcloud={"zoom_width":100}