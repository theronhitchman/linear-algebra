\documentclass[11pt]{amsart}
\usepackage[margin=1in]{geometry}
\usepackage{paralist}
\usepackage{graphicx}
\usepackage{mathtools}

\theoremstyle{definition}
\newtheorem{exercise}{Exercise}
\newtheorem*{theorem}{Theorem}

\begin{document}
\title{Linear Algebra}
\author{Strang, Section 3.3}
\maketitle

\section{The assignment}
\begin{compactitem}
\item Read section 3.3 of Strang (pages 144-151).
\item Read the following and complete the exercises below.
\end{compactitem}


\section{The Rank of a Matrix}

The new concept for this section is the \emph{rank} of a matrix. There are \textbf{three} definitions of rank given in the first two pages of this section. It is important to know them and see that they are equivalent.

Another important part of what happens here is the realization that putting the matrix into reduced row echelon form, which uses \textbf{row} operations, actually tells us something about the \textbf{columns} of the matrix. This is kind of amazing.

Take a matrix $A$ and put it into reduced row echelon form $R = \mathrm{rref}(A)$. Then $A$ and $R$ have the same shape. In particular, they have the same number of columns. But in $R$ it is easy to see the location of the pivots. This divides up the columns of $R$ into those columns having a pivot, the \emph{pivot columns}, and those which do not, the \emph{free columns}.
We will label a column of $A$ as a pivot column or a free column in a way that corresponds.
Here is the startling fact:

\begin{theorem}
Let $A$ be a matrix.
The pivot columns of $A$ are those which cannot be expressed as linear combinations of the columns to their left.
The free columns of $A$ are those which can be expressed as linear combinations of the columns to their left.
\end{theorem}

How exactly can we see this? Strang uses the construction of a \emph{nullspace matrix}. The special solutions we learned about in the last section can be bundled together as the columns of a matrix $N$, called the nullspace matrix. This matrix has two important properties:
\begin{itemize}
\item $AN = 0$; and
\item Each column of $N$ (i.e. each special solution) holds the coefficients needed to write down a interesting linear combination equation on the columns of $A$.
\end{itemize}

Finally, please use some caution when reading through the area with blue boxes and equations (4) and (5) on page 147. Note that Strang introduces a big simplifying assumption that makes his work easier. The general principles will hold, but those nice, neat equations won't always look so good for an arbitrary matrix.


\section{Sage instructions}

I have made a Sage worksheet file with some basic commands that you might find useful in investigating matrices. The file is called \texttt{section3\_3.sagews}.


\section{Questions for Section 3.3}
\setcounter{exercise}{89}

\begin{exercise}
Give an example of a $5 \times 3$ matrix which has
\begin{compactitem}
\item rank equal to 3, and
\item no non-zero entries,
\end{compactitem}
or explain why no such example is possible.
\end{exercise}

\begin{exercise}
Given an example of a $3 \times 3$ matrix which has
\begin{compactitem}
\item rank equal to 2,
\item a nullspace of $\{0\}$, and
\item no non-zero entries,
\end{compactitem}
or explain why no such example is possible.
\end{exercise}


\begin{exercise}
Consider the matrix $A = \left( \begin{smallmatrix} 2 & 1 & 5 \\ 1 & 1 & 7 \end{smallmatrix}\right)$. Find the reduced row echelon form $R = \mathrm{rref}(A)$ of $A$. Track the row operations you use, and use them to find an invertible matrix $E$ so that $EA = R$.
\end{exercise}

\begin{exercise}
Continuing the last exercise\dots Find the null space matrix of $A$. Use the information contained in the nullspace matrix to write down a linear combination equation on the columns of $R = \mathrm{rref}(A)$ of the form
\[
a * \text{(column1)} + b * \text{(column2)} + c * \text{(column3)} = 0.
\]

Explain why the matrix $E$ allows us to translate this equation into this one on the columns of $A$:
\[
a\begin{pmatrix} 2 \\ 1 \end{pmatrix} + b \begin{pmatrix} 1 \\ 1 \end{pmatrix} + c \begin{pmatrix} 5 \\ 7 \end{pmatrix} = \begin{pmatrix}0\\ 0 \end{pmatrix}.
\]
\end{exercise}

\begin{exercise}
Consider the matrix $N$ given below. Make an example of two different matrices $A$ and $B$ which have different shapes and each have $N$ as nullspace matrix, or explain why such an example is not possible.
\[
N = \begin{pmatrix}
-3 & 2 & -1 \\
 1 & 0 &  0 \\
 0 & 1 &  0 \\
 2 & 6 &  1 \\
 0 & 0 &  1
\end{pmatrix}
\]

\end{exercise}

\begin{exercise}
Consider the matrix $T = \left(\begin{smallmatrix} 2 & 1 \\ 1 & 1 \end{smallmatrix}\right)$. What is the nullspace matrix of $T$?
\end{exercise}



\end{document}




%sagemathcloud={"zoom_width":100}