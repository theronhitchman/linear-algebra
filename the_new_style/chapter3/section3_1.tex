\documentclass[11pt]{amsart}
\usepackage[margin=1in]{geometry}
\usepackage{paralist}
\usepackage{graphicx}
\usepackage{mathtools}

\theoremstyle{definition}
\newtheorem{exercise}{Exercise}

\begin{document}
\title{Linear Algebra}
\author{Strang, Section 3.1}
\maketitle

\section{The assignment}
\begin{compactitem}
\item Read section 3.1 of Strang (pages 120-127).
\item Read the following and complete the exercises below.
\end{compactitem}


\section{Vector Spaces and Subspaces}

The key concepts in this section are those of a \emph{vector space} and of a \emph{subspace}. The basic idea is that a vector space is a kind of place where the basic operations involved in a linear combination make sense. There is a set of rules for being a vector space, but they are all aimed at the fact that there are two operations  (addition and scalar multiplication) and we can form linear combinations with them.

The biggest thing is that we possibly enlarge the kinds of things we call ``vectors.'' My favorites are things like this:
\begin{compactitem}
\item The set $M_{2,2}$ of $2\times 2$ matrices is a vector space. But now the things we call vectors are actually matrices.

\item The set $\mathcal{C}(\mathbb{R})$ of continuous functions with domain and range both equal to the set of real numbers is a vector space. But now the things we call vectors are actually functions.

\item The set $\ell(\mathbb{R})$ of sequences $(x_1, x_2, x_3, x_4, \ldots)$ of real numbers is a vector space. But now the things we call vectors are actually sequences.
\end{compactitem}


The idea of a subspace is some subset, some part, of a vector space which is a vector space in its own right. The prototype is the $xy$-plane inside of $\mathbb{R}^3$.


For now, the most important subspaces we see will be derived from individual matrices. Our first example is the column space of a matrix $A$. If $A$ is an $m\times n$ matrix, then the column space $\mathrm{col}(A)$ of $A$ is the collection of $n$-vectors which can be expressed as linear combinations of the columns of $A$. This is our first exposure to the idea of a \emph{span}. The column space of $A$ is the subspace of $\mathbb{R}$ spanned by the columns of $A$.


\section{Sage instructions}

I have made a Sage worksheet file with some basic commands that you might find useful in investigating linear systems. The file is called \texttt{section3\_1.sagews}.


\section{Questions for Section 3.1}
\setcounter{exercise}{77}

\begin{exercise}
Find an example of a vector which is not in the column space of the matrix $$A = \begin{pmatrix} 2 & 1  \\ 1 & 1 \end{pmatrix},$$ or explain why that is not possible.
\end{exercise}

\begin{exercise}
Find an example of a vector which is not in the column space of the matrix $$B = \begin{pmatrix} 3 & 2 \\ 6 & 4 \end{pmatrix},$$ or explain why it is not possible.
\end{exercise}

\begin{exercise}
Let $\mathcal{P}$ be the set of polynomials of degree 3 or less. Explain why $\mathcal{P}$ is a vector space, or explain why it is not.
\end{exercise}

\begin{exercise}
Consider the vector space $\mathbb{R}^2$. Explain why the following are not subspaces:
\begin{compactitem}
\item The unit circle.
\item The line $x+y = 4$.
\item The union of lines $2x+3y = 0$ and $x-y=0$.
\item The first quadrant where $x\geq 0$ and $y\geq 0$.
\end{compactitem}
\end{exercise}

\begin{exercise}
Let $W$ be the set of functions $f$ in $\mathcal{C}(\mathbb{R})$ which satisfy the differential equation $f''(x) = 0$. Decide if $W$ is a subspace of $\mathcal{C}(\mathbb{R})$, and explain your thinking.
\end{exercise}

\begin{exercise}
Design a $3\times 2$ matrix whose column space does not contain the vector $\begin{pmatrix} 4 \\ 4 \end{pmatrix}$.
\end{exercise}

\end{document}




%sagemathcloud={"zoom_width":100}