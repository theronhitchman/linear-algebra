\documentclass[11pt]{amsart}
\usepackage[margin=1in]{geometry}
\usepackage{paralist}
\usepackage{graphicx}
\usepackage{mathtools}

\theoremstyle{definition}
\newtheorem{exercise}{Exercise}

\begin{document}
\title{Linear Algebra}
\author{Strang, Section 3.2}
\maketitle

\section{The assignment}
\begin{compactitem}
\item Read section 3.2 of Strang (pages 132-140).
\item Read the following and complete the exercises below.
\end{compactitem}


\section{The Nullspace, RREF, \& Special Solutions to $Ax=0$}

Strang aims right at the heart of things in this section, and does not waste any space. Let me highlight a few things:

Let $A$ be an $m\times n$ matrix. We do not require that $A$ is a square matrix, that is, it may not be the case that $m=n$. What we are most interested in is solving the equation $Ax = 0$. Note that if $A$ is not square, then the vectors $x$ and $0$ have different sizes.

By the way, an equation like $Ax=0$ where the right hand side is zero is called a \emph{homogeneous equation}.

\begin{itemize}

\item The nullspace of $A$ is the set of vectors $x$ such that $Ax=0$. It is a theorem that this is a vector subspace of $\mathbb{R}^n$. It is a common to use the synonym \emph{the kernel of $A$} in place of the terminology \emph{the nullspace of $A$.}

\item The key to everything is to not mind that your matrix isn't square. Just do Gauss-Jordan elimination anyway. The end result is called the \emph{reduced row echelon form} of the matrix, or RREF for short.

\item Note that there is no discussion of using an augmented matrix in this section, even though we are solving systems of equations $Ax=0$. This is because the vector on the right hand side is zero and will stay zero. There is nothing interesting to track!

\item The RREF is good for several things. One that is often overlooked is that one can use it to rewrite the system of equations in a ``simplest form.'' I mean that the equations left over in the RREF are somehow the easiest equations to use to cut out the solution as an intersection of hyperplanes.

\item In the RREF, all of the structure can be inferred from the splitting of the columns into two types: the pivot columns and the free columns. Since each column is associated to a variable in our system of linear equations (the columns hold coefficients!), it is also common to refer to pivot variables and free variables.

\item The number of free columns basically determines the ``size'' of the nullspace. This is an entry point to the concept of the \emph{dimension} of a vector space. We shall see this in more detail later.

\item Strang points out an easy way to find some individual vectors in the null space: he calls these the \emph{special solutions}. This is because they are solutions to the equation $Ax=0$.

\end{itemize}


\section{Sage instructions}

I have made a Sage worksheet file with some basic commands that you might find useful in investigating linear systems. The file is called \texttt{section3\_2.sagews}.


\section{Questions for Section 3.2}
\setcounter{exercise}{83}

For the first three exercises, your job is to find both
\begin{compactitem}
\item The minimal set of equations which cuts out the nullspace of the given matrix as an intersection of hyperplanes; and

\item the size of the nullspace, expressed through the number ``independent directions'' it contains.
\end{compactitem}
As always, it will be crucial to explain how you know you are correct.

\begin{exercise}
Consider the matrix
\[
A = \begin{pmatrix}
1 & 2 & 3 \\
4 & 8 & 5 \\
-1 & -2 & 0
\end{pmatrix}
\]
\end{exercise}

\begin{exercise}
Consider the matrix
\[
A = \begin{pmatrix}
2 & 1 & 3 & 7 \\
1 & 1 & 1 & -3
\end{pmatrix}
\]
\end{exercise}

\begin{exercise}
Consider the matrix
\[
A = \begin{pmatrix}
2 & 3 & 5 & 6 \\
1 & 1 & 2 & 3 \\
0 & 1 & 1 & 0 \\
-1 & -1 & -2 & -3 \\
1 & 1 & 2 & 3
\end{pmatrix}
\]
\end{exercise}

For the next three exercises, your job is to find a complete set of special solutions to the homogeneous equation $Ax=0$. By a ``complete set,'' we mean ``enough special solutions so that any vector in the nullspace of $A$ can be expressed as a linear combination of your vectors.''

\begin{exercise}
Consider the matrix
\[
A = \begin{pmatrix}
6 & 7 \\
7 & 8 \\
1 & 0 \\
4 & 5
\end{pmatrix}
\]
\end{exercise}

\begin{exercise}
Consider the matrix
\[
A = \begin{pmatrix}
23 & 17 & 9 & 2 \\
1 & -2 & 4 & 1 \\
22 & 19 & 5 & 1
\end{pmatrix}
\]
\end{exercise}

\begin{exercise}
Consider the matrix
\[
A = \begin{pmatrix}
-3 & -6 & -9 & -12 \\
1 & 2 & 3 & 4 \\
1 & 4 & 9 & 16 \\
1 & 8 & 27 & 64 \\
-1 & -1 & -1 & -1 \\
0 & 0 & 1 & 0 
\end{pmatrix}
\]
\end{exercise}
As always, it will be crucial to explain how you know you are correct.


\end{document}




%sagemathcloud={"zoom_width":100}