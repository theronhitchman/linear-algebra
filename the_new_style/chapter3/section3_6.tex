\documentclass[11pt]{amsart}
\usepackage[margin=1in]{geometry}
\usepackage{paralist}
\usepackage{graphicx}
\usepackage{mathtools}

\theoremstyle{definition}
\newtheorem{exercise}{Exercise}
\newtheorem*{theorem}{Theorem}

\begin{document}
\title{Linear Algebra}
\author{Strang, Section 3.6}
\maketitle

\section{The assignment}
\begin{compactitem}
\item Read section 3.6 of Strang (pages 184-190).
\item Read the following and complete the exercises below.
\end{compactitem}


\section{The Four Subspaces}

This section summarizes a big tool for understanding the behavior of a matrix as a function.
Recall that if $A$ is an $m\times n$ matrix, then we can think of it as defining a function
\[
\begin{array}{rcl}
T_A: \mathbb{R}^n & \rightarrow & \mathbb{R}^m \\
v & \mapsto & Av
\end{array}
\]
which takes as inputs vectors from $\mathbb{R}^n$ and has as outputs vectors in $\mathbb{R}^m$. We have also seen that properties of matrix multiplication translate into properties that make into a \emph{linear transformation}.

We now have four fundamental subspaces associated to the matrix $A$.
\begin{itemize}
\item The column space, $\mathrm{col}(A)$, spanned by all of the columns of $A$. This is a subspace of $\mathbb{R}^m$.
\item The row space, $\mathrm{row}(A)$, spanned by all of the rows of $A$. This is a subspace of $\mathbb{R}^n$. This also happens to be the column space of $A^T$.
\item The nullspace (or kernel), $\mathrm{null}(A)$, consisting of all those vectors $x$ for which $Ax=0$. This is a subspace of $\mathbb{R}^n$.
\item The left nullspace, which is just the nullspace of $A^T$. This is a subspace of $\mathbb{R}^m$.
\end{itemize}

And we have a big result:

\begin{theorem}
If $A$ is an $m\times n$ matrix with rank $\mathrm{rank}(A) = r$, then
\begin{compactitem}
\item $\dim(\mathrm{col}(A)) = \dim(\mathrm{row}(A)) = r$,
\item $\dim(\mathrm{null}(A)) = n-r$, and
\item $\dim(\mathrm{null}(A^T))= m-r$.
\end{compactitem}
\end{theorem}

(\emph{Study Hint: Write that out in English, with no notation. It will help you remember it.}

We will have more to say about these spaces when we reconsider the uses of the dot product in chapter 4.

\section{Sage instructions}

I have made a Sage worksheet file with some basic commands that you might find useful in investigating matrices. The file is called \texttt{section3\_6.sagews}.


\section{Questions for Section 3.6}
\setcounter{exercise}{110}

Note: We may not present all of these in class.

\begin{exercise} Find the four subspaces, including a basis of each, for the matrix
\[
A = \begin{pmatrix}
7 & -1 & 3 \\
-2 & 4 & -5 \\
1 & 11 & -12
\end{pmatrix}.
\]
\end{exercise}
\begin{exercise} Find the four subspaces, including a basis of each, for the matrix
\[
B = \begin{pmatrix}
1 & 3 & -2 & 0 & 2 & 0 \\
2 & 6 & -5 & -2 & 4 & -3 \\
0 & 0 & 5 & 10 & 0 & 15 \\
2 & 6 & 0 & 8 & 4 & 18
\end{pmatrix}.
\] \end{exercise}
\begin{exercise} Do Strang Exercise 3.6 number 12. \end{exercise}
\begin{exercise} Do Strang Exercise 3.6 number 14. \end{exercise}
\begin{exercise} Do Strang Exercise 3.6 number 16. \end{exercise}
\begin{exercise} Do Strang Exercise 3.6 number 24. \end{exercise}


\end{document}




%sagemathcloud={"zoom_width":100}