\documentclass[11pt]{amsart}
\usepackage[margin=1in]{geometry}
\usepackage{paralist}
\usepackage{graphicx}
\usepackage{mathtools}

\theoremstyle{definition}
\newtheorem{exercise}{Exercise}
\newtheorem*{theorem}{Theorem}

\begin{document}
\title{Linear Algebra}
\author{Strang, Section 3.5}
\maketitle

\section{The assignment}
\begin{compactitem}
\item Read section 3.5 of Strang (pages 168-178).
\item Read the following and complete the exercises below.
\end{compactitem}


\section{Linear Independence, Spanning, Basis, and Dimension}

The purpose of this lesson is to introduce specific terms for several concepts we have been dancing around. This is a spot that sometimes gives students difficulty because they are unused to the way mathematicians talk. So, here is a big warning:
\begin{quotation}
"When I use a word," Humpty Dumpty said in rather a scornful tone, "it means just what I choose it to mean -- neither more nor less."

\hfill\emph{Through the Looking Glass}, Lewis Carroll
\end{quotation}
That is the essence of it. We have several new words, and they will mean \emph{exactly} what we declare them to mean, no more, no less.

What are the new terms?
\begin{compactitem}
\item A set $\{v_1, v_2, \ldots, v_k\}$ of vectors is linearly independent, or linearly dependent.
\item A set of vectors spans a vectors space or a subspace.
\item A set of vectors is a basis or not.
\item The dimension of a vector space, or of a vector subspace.
\end{compactitem}
You should take away from this reading what those four terms are, how to check them, and some examples and non-examples.


A bit of notation: If we have a set $\{ v_1, v_2, \dots, v_k\}$ of vectors in some vector space, then we denote the subspace which they span by $\mathrm{span}(\{v_1, v_2, \ldots, v_k\})$. (This has to be the easiest possible notational choice ever.)

\subsection{Other Vector Spaces}

Once you understand these terms as they apply to the Euclidean spaces $\mathbb{R}^n$ and their subspaces (especially those associated to matrices), you should pause to admire your achievement.

But next, realize that those terms apply generally to all sorts of vector spaces! Can you make examples and non-examples in some of these other situtations?
\begin{compactitem}
\item The set $\mathcal{P}_3$ of polynomials in $x$ of degree no more than 3?
\item The set $M_{m,n}$ of $m\times n$ matrices.
\item The set $\mathcal{C}(\mathbb{R})$ of continuous functions.
\item The set of all functions which are solutions to the differential equation $y'' = y$.
\end{compactitem}


\subsection{Two Methods of Sorting out Linear Independence}

We have enough information to collect two ways to answer the question: "Is this set linearly independent?"
Well, at least when working with vectors in some Euclidean space $\mathbb{R}^n$.

\subsubsection{The Column Space Algorithm}
Given a set of vectors $\{v_1, v_2, \ldots, v_k\}$ from $\mathbb{R}^n$.
\begin{compactenum}
\item Form the $n\times k$ matrix $A = \left( \begin{smallmatrix} v_1 & v_2 & \dots & v_k\end{smallmatrix}\right)$.
\item Put $A$ into reduced row echelon form $R = \mathrm{rref}(A)$. (Really, you only need to go to echelon form, here.)
\item Read out pivot columns and free columns of $R$. Those columns of $A$ which are free columns are linear combinations of previous columns to the left! So, if any column of $R$ (and $A$) is a free columns, the set of vectors is linearly dependent. If all of the columns of $R$ (and $A$) are pivot columns, then the set is linearly independent.
\end{compactenum}

This method is particularly good for identifying which subset of our original set of vectors would form a basis of the vector subspace $\mathrm{span}(\{v_1, v_2, \ldots, v_k\})$.


\subsubsection{The Row Space Algorithm}
(We will see the reason for the name soon.)
Given a set of vectors $\{v_1, v_2, \ldots, v_k\}$ from $\mathbb{R}^n$.
\begin{compactenum}
\item Form the $k\times n$ matrix $A = \left( \begin{smallmatrix} v_1 \\ v_2 \\ \vdots \\ v_k\end{smallmatrix}\right)$.
\item Put $A$ into reduced row echelon form $R = \mathrm{rref}(A)$.
\item The rows of $R$ will contain a basis for the vector subspace $\mathrm{span}(\{v_1, v_2, \ldots, v_k\})$. If $R$ has any zero rows, the original set was linearly dependent, otherwise it was linearly independent.
\end{compactenum}

This method is good at picking out a simple basis of the vector subspace $\mathrm{span}(\{v_1, v_2, \ldots, v_k\})$, but the resulting vectors \emph{probably won't come from your original set}.

\section{Sage instructions}

I have made a Sage worksheet file with some basic commands that you might find useful in investigating matrices. The file is called \texttt{section3\_5.sagews}.


\section{Questions for Section 3.5}
\setcounter{exercise}{103}

Note: We may not present all of these in class.

\begin{exercise}
Make an example of a vector $v \in \mathbb{R}^4$ so that the set
\[
\left\{
\begin{pmatrix} 1 \\ 2 \\ 3 \\ 4 \end{pmatrix},
\begin{pmatrix} 0 \\ 1 \\ 2 \\ 1 \end{pmatrix},
\begin{pmatrix} -2\\ 1 \\ 0 \\ -5 \end{pmatrix},
v
\right\}
\]
is a linearly independent set, or explain why it is impossible to find such an example.
Is your resulting set a basis?
\end{exercise}

\begin{exercise}
Make an example of a vector $w \in \mathbb{R}^3$ so that the set
\[
\left\{
\begin{pmatrix} 2 \\ 3 \\ 4 \end{pmatrix},
\begin{pmatrix} 0 \\ 1 \\ 1 \end{pmatrix},
\begin{pmatrix} -2\\ -4 \\ -5 \end{pmatrix},
w
\right\}
\]
is a spanning set, or explain why it is impossible to find such an example. Is your resulting set a basis?
\end{exercise}

\begin{exercise} Do section 3.5 exercise 10 from \emph{Strang}.\end{exercise}
\begin{exercise} Do section 3.5 exercise 21 from \emph{Strang}.\end{exercise}
\begin{exercise} Do section 3.5 exercise 22 from \emph{Strang}.\end{exercise}
\begin{exercise} Do section 3.5 exercise 26 from \emph{Strang}.\end{exercise}
\begin{exercise} Do section 3.5 exercise 35 from \emph{Strang}.\end{exercise}


\end{document}




%sagemathcloud={"zoom_width":100}