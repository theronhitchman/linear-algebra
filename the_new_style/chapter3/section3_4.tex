\documentclass[11pt]{amsart}
\usepackage[margin=1in]{geometry}
\usepackage{paralist}
\usepackage{graphicx}
\usepackage{mathtools}

\theoremstyle{definition}
\newtheorem{exercise}{Exercise}
\newtheorem*{theorem}{Theorem}

\begin{document}
\title{Linear Algebra}
\author{Strang, Section 3.4}
\maketitle

\section{The assignment}
\begin{compactitem}
\item Read section 3.4 of Strang (pages 155-162).
\item Read the following and complete the exercises below.
\end{compactitem}


\section{The General Solution to a System of Linear Equations}

This is the big day! We finally learn how to write out the general solution to a system of linear equations. We have spent so much time understanding things related to this, that it should go pretty quickly.

The tiny little facts underneath the analysis for this section are these:
For a matrix $A$, vectors $v$ and $w$ and a scalar $\lambda$, all chosen so that the equations make any sense,
\[
\begin{array}{rcl}
A(v+w) & = & Av + Aw \\
A(\lambda v) & = & \lambda ( Av )
\end{array}
\]

The first is a kind of \emph{distributive property}, and the second is a kind of \emph{commutative property}. When taken together, these things say that the operation of ``left-multiply by the matrix $A$'' is a special kind of function. The kind of function here is important enough that we have a special word for this combined property: it is called \emph{lineararity}. That is, left-multiplication by $A$ is a \emph{linear operation} or a \emph{linear transformation}.

The linearity property makes it easy to check the following two results.
\begin{theorem}
Let $Ax=b$ be a system of linear equations, and let $Ax=0$ be the associated homogeneous system.

If $x_p$ and $x_p'$ are two particular solutions to $Ax=b$, then $x_p - x_p'$ is a solution to the homogeneous system $Ax=0$.
\end{theorem}

\begin{theorem}
Let $Ax=b$ be a system of linear equations, and let $Ax=0$ be the associated homogeneous system.

If $x_p$ is some particular solution to $Ax=b$ and $x_n$ is some solution to $Ax=0$, then $x_p + x_n$ is another solution to $Ax=b$.
\end{theorem}

And if we put these two theorems together, we find this result which sounds fancier, but has exactly the same content.
\begin{theorem}
The complete set of solutions to the system $Ax=b$ is the set $\left\{ x_p + x_n \mid x_n \in \mathrm{null}(A) \right\}$,

where $x_p$ is any one particular solution to $Ax=b$.
\end{theorem}


This leads us to Strang's very sensible advice about finding the complete solution:
\begin{compactitem}
\item Form the augmented matrix $\left( A \mid b \right)$ and use Gauss-Jordan elimination to put it in reduced row echelon form $\left( R \mid d \right)$.

\item Use the information from the RREF to find a particular solution $x_p$ by solving for the pivot variables from the vector $d$ and setting the free variables to zero.

\item Use the special solutions $s_1, s_2, \dots, s_k$ (if any exist!) to describe the nullspace $\mathrm{null}(A)$.

\item Write down the resulting general solution:
\[
x = x_p + a_1 s_1 + a_2 s_2 + \dots + a_k s_k, \quad a_i \in \mathbb{R}.
\]
\end{compactitem}

\section{Sage instructions}

I have made a Sage worksheet file with some basic commands that you might find useful in investigating matrices. The file is called \texttt{section3\_3.sagews}.


\section{Questions for Section 3.4}
\setcounter{exercise}{95}

Note: We may not present all of these in class.

\begin{exercise} Do Strang Exercise 3.4 number 4. \end{exercise}
\begin{exercise} Do Strang Exercise 3.4 number 6. \end{exercise}
\begin{exercise} Do Strang Exercise 3.4 number 11. \end{exercise}
\begin{exercise} Do Strang Exercise 3.4 number 13. Find simple examples which show each statement is false. \end{exercise}
\begin{exercise} Do Strang Exercise 3.4 number 21. \end{exercise}
\begin{exercise} Do Strang Exercise 3.4 number 24. \end{exercise}
\begin{exercise} Do Strang Exercise 3.4 number 31. \end{exercise}
\begin{exercise} Do Strang Exercise 3.4 number 33. \end{exercise}


\end{document}




%sagemathcloud={"zoom_width":100}