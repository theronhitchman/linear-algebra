\documentclass[11pt]{amsart}

\usepackage[margin=1in]{geometry}
\usepackage{paralist}

\theoremstyle{definition}
\newtheorem{task}{Task}
\newtheorem{question}[task]{Question}

\begin{document}
\title{Linear Algebra Midterm: Spring 14}
\author{Section 01: Hitchman}
\date{}

\maketitle

\noindent
\textbf{Instructions:} Please write your answers on the blank paper provided.
Be sure your name is on each sheet of paper.
Explain your thinking clearly and in complete sentences.
This examination has 13 questions on 2 pages.

\section{Basic Computational Fluency}

For most of this section, no work is required. No partial credit will be given. Just write down the correct outcome from the computation.



\begin{task} Add the two vectors
\begin{equation*}
\begin{pmatrix} 3 \\ 4 \end{pmatrix} \text{ and }
\begin{pmatrix} -5 \\ 2 \end{pmatrix}.
\end{equation*}
\end{task}

\begin{task} Add the two matrices
\begin{equation*}
\begin{pmatrix} 3 & 4 & 7 \\ 1 & -1 & -2 \end{pmatrix} \text{ and }
\begin{pmatrix} -3 & 0 & 3 \\ 1 & 2 & 6 \end{pmatrix}.
\end{equation*}
\end{task}

\begin{task} Write down the transpose of each of these two matrices:
\begin{equation*}
A = \begin{pmatrix} 6 & 1 & 2 \end{pmatrix} \text{ and }
B = \begin{pmatrix} 1 & 2 \\ 3 & 4 \\ 5 & 6 \end{pmatrix}.
\end{equation*}
\end{task}

\begin{task} Find the product of these two matrices in the order that makes sense:
\begin{equation*}
C = \begin{pmatrix} 5 & 1 & 0 \\ 0 & 1 & -1 \end{pmatrix} \text{ and }
D = \begin{pmatrix} 2 & 1 \\ 1 & 1 \end{pmatrix}
\end{equation*}
\end{task}

\begin{task} Give an example of a pair of $2\times 2$ matrices $X$ and $Y$ which do not commute.
\end{task}

\begin{task} Compute this linear combination of vectors
\begin{equation*}
5 \begin{pmatrix} 2 \\ 1 \end{pmatrix} - 7 \begin{pmatrix} 3 \\ 2 \end{pmatrix}.
\end{equation*}
\end{task}

\begin{task} Compute the dot product of the vectors
\begin{equation*}
\begin{pmatrix} 2 \\ 1 \end{pmatrix} \text{ and } \begin{pmatrix} 3 \\ 2 \end{pmatrix}.
\end{equation*}
\end{task}

\begin{task} Compute the norm of the vector
\begin{equation*}
\begin{pmatrix} 2 \\ 1 \end{pmatrix}.
\end{equation*}
\end{task}


\begin{task} Compute the angle between the vectors
\begin{equation*}
\begin{pmatrix} 2 \\ 1 \end{pmatrix} \text{ and } \begin{pmatrix} 3 \\ 2 \end{pmatrix}.
\end{equation*}
(Write an exact expression that gives the angle. Do not write a decimal approximation.)
\end{task}





\section{Interpretations}

The next few tasks ask for careful translations between our different viewpoints. Describe yourself clearly.

\begin{task} We have seen two ways to compute the product below, which involes multiplying a matrix times a vector. Describe them both briefly, and show that they give the same result.
\begin{equation*}
\begin{pmatrix}
2 & 1 \\ 1 & 1
\end{pmatrix}
\begin{pmatrix} -1 \\ 3 \end{pmatrix}
\end{equation*}
\end{task}

\begin{task} We are given a situation where the unknown vector $X = \left(\begin{smallmatrix} x \\ y \\ z \end{smallmatrix} \right)$ is perpendicular to each of the vectors $U = \left( \begin{smallmatrix} -3 \\ 0 \\ 1 \end{smallmatrix} \right)$ and $V = \left(\begin{smallmatrix} 4 \\ 4 \\ 0 \end{smallmatrix}\right)$. Write these conditions as a system of linear equations on the components of $X$.
\end{task}



\begin{task} We have seen that a system of linear equations can be written in two other algebraic forms involving such things as matrices or vectors. For the system below, write down those alternate forms. (Don't solve the system of equations. Just give the other forms.)
\begin{equation*}
\left\{\begin{array}{rrrrrrr}
2x & + & 3y & - & z & = & 7\\
4x & - & y & + & 2z & = & 0
\end{array}\right.
\end{equation*}
\end{task}



\section{Gauss-Jordan Elimination \& Matrix Factorization}

This last section asks you to show you understand the basics of Gauss-Jordan Elimination. Be sure to show your work.

\begin{task}
Consider the matrix $H$ below.
\begin{equation*}
H = \begin{pmatrix} 2 & 1 & 6 \\ 1 & 0 & 3 \\ 0 & 0 & 3 \end{pmatrix}
\end{equation*}
Use Gauss-Jordan elimination to find
\begin{compactitem}
\item The LU decomposition of $H$, and
\item The inverse of $H$.
\end{compactitem}
\end{task}

%\begin{task}
%Let $H$ be the matrix given in the last task. Let $b$ be the vector
%\begin{equation*}
%b = \begin{pmatrix} 1 \\ 0 \\ 1 \end{pmatrix}
%\end{equation*}
%Use the LU decomposition of $H$ to solve the equation $Hx = b$ by solving two triangular systems with the back-substitution technique.
%\end{task}



\end{document}
%sagemathcloud={"zoom_width":100}