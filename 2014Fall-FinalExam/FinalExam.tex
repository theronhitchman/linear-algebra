\documentclass[11pt]{amsart}
\usepackage[margin=1in]{geometry}
\usepackage{paralist}

\theoremstyle{definition}
\newtheorem{problem}{Problem}[section]

\begin{document}

\title{Final Exam}
\author{Math 2500, Hitchman}
\date{May 7, 2014}

\maketitle

\section*{Instructions}

You may use a calculator for this exam. Alternately, you may use an internet connected device to access the Sage Cell Server, or SageMathCloud to do computations.

Please write your responses on the blank paper provided. Put your name on each sheet, and staple all the sheets together when you turn them in.

Unless noted, each section of the exam is worth 10 points. The whole exam will be scored out of 100 points, though more are available.



\section{Matrix and Vector Algebra, the dot product}

\begin{problem}
We have discussed, and found good use for, two different ways to think about computing the product of a matrix $A$ with a vector $v$ like so: $Av$. What are these two ways?
\end{problem}

\begin{problem}
What is the dot product, and why have we found it important?
\end{problem}


\vspace{1cm}
\section{The three pictures}

\begin{problem}
We have discussed three separate ways of writing a system of $m$ linear equations in $n$ unknowns. Each has a geometric interpretation. What are these ways? For each, describe what it means geometrically to find a solution to the system of linear equations.
\end{problem}

\vspace{1cm}
\section{Solving a system of linear equations}

\begin{problem}
Find the complete solution to the following system of linear equations written in matrix-vector form:
\[
\begin{pmatrix} 2 & 4 & 6 & 4\\ 2 & 5 & 7 & 6 \\ 2 & 3 & 5 & 2 \end{pmatrix}
\begin{pmatrix} w \\ x \\ y \\ z \end{pmatrix}
= \begin{pmatrix} 4 \\ 3 \\ 5 \end{pmatrix}
\]
\end{problem}

\pagebreak



\section{The $LU$ decomposition and Gauss-Jordan Elimination}

\begin{problem}
How does the $LU$ decomposition of a square matrix connect to the algebra of Gauss-Jordan Elimination?
\end{problem}

\begin{problem}
Suppose that $B$ is an $m\times n$ matrix where $m \neq n$. So $B$ is \emph{not square}. How does the process of putting $B$ into reduced row echelon form translate into matrix language? Name something interesting we can learn from the resulting matrix decomposition.
\end{problem}

\vspace{1cm}
\section{The Four Subspaces}

\begin{problem}
What are the four subspaces associated to a matrix?
\end{problem}

\begin{problem}
The \emph{Fundamental Theorem of Linear Algebra} tells us some special things about the four subspaces. Say what it tells us.
\end{problem}


\begin{problem}
Suppose that $A$ is a $2\times 3$ matrix with rank equal to $1$. Draw a picture which illustrates what the four subspaces are in such a case and how they relate to each other.
\end{problem}


\vspace{1cm}
\section{Bases}

\begin{problem}
What does it mean to say that a set of vectors is a basis for some subspace?
Give examples of things that are easily mistaken for a basis, but are not.
\end{problem}


\vspace{1cm}
\section{Describing hyperplanes: parametric vs. implicit}

\begin{problem}
Let $S$ be the subspace of $\mathbb{R}^4$ which is cut out by these equations
\[
\begin{array}{rrrrrrrrr}
3w & -& 2x & + & y & + & z & = & 0 \\
   &  & x  & + & y & - & z & = & 0
\end{array}
\]
Describe a basis for this subspace. How does a basis help you describe the subspace $S$ \emph{parametrically}?
\end{problem}


\begin{problem}
Now let $V$ be the subspace of $\mathbb{R}^4$ with a basis $\beta = \{ v_1, v_2 \}$ where
\[
v_1 = \begin{pmatrix} -1\\ 0 \\ 1 \\ 3 \end{pmatrix} \qquad v_2 = \begin{pmatrix} 0 \\ 1 \\ 0 \\ 2 \end{pmatrix}.
\]
Find a set of linear equations which determine whether a vector $(w,x,y,z) \in \mathbb{R}^4$ lies in $V$ or not.
\end{problem}



\pagebreak

\section{Determinants and the invertible matrix theorem}

\begin{problem}
Suppose that $A$ is a square matrix. We have seen many conditions which are equivalent to the fact $\det(A) \neq 0$. Give three such conditions.
\end{problem}

\begin{problem}
Suppose you have a matrix $A$ already decomposed into $A = LU$ form. How would you compute the determinant?

Then, make an example of $A= LU$ which is $3\times 3$ and which shows how the determinant can be zero. You need not multiply out to find $A$, just the form $LU$ as a product of matrices is okay.
\end{problem}


\vspace{1cm}
\section{Geometry, Projections, Least Squares}

\begin{problem}
Find the matrix which projects vectors in $\mathbb{R}^3$ onto the plane $x - y + z = 0$.

(I don't actually need to see the final projection matrix itself, though I will take that. But you must convince me that you know how to find it with only simple computational steps remaining.)
\end{problem}

\begin{problem}
How is the technique of ``least squares'' related to the geometry of projections?
\end{problem}

\vspace{1cm}
\section{The $QR$ decomposition}

\begin{problem}
Find the $QR$ decomposition of $\left( \begin{smallmatrix} 2 & 1 \\ 1 & 1 \end{smallmatrix}\right)$. (I don't care if you do it by hand or not.)
\end{problem}

\begin{problem}
The computation from the last problem should mean something about two different bases of the plane. Draw a picture of this and describe what happened.
\end{problem}

\vspace{1cm}
\section{Eigenvalues and Eigenvectors}

\begin{problem}
Suppose that $A$ is a square matrix. Describe the process for finding the eigenvalues of $A$.
\end{problem}

\begin{problem}
Now, we have a square matrix $A$ and its eigenvalues $\lambda_1, \ldots, \lambda_k$.
How do you find the eigenvectors of $A$?
\end{problem}

\begin{problem}
Write down a $4\times 4$ matrix which has the following eigenvalues and eigenvectors:
\begin{compactitem}
\item $\lambda_1 = 3$ has corresponding eigenvectors
\[
\begin{pmatrix} 2 \\ 1\\ -1\\ 1 \end{pmatrix}, \qquad \begin{pmatrix} 3\\ 1\\ 0\\0 \end{pmatrix}, \qquad \text{and } \begin{pmatrix} 0\\ -1\\ 3\\ 0 \end{pmatrix}
\]
\item $\lambda_2 = -1$ has corresponding eigenvector
\[
\begin{pmatrix} 0 \\ 0 \\ -1 \\ 1 \end{pmatrix}.
\]
\end{compactitem}
(I don't actually need to see the final $4\times 4$ matrix itself, though I will take that. But you must convince me that you know how to find it with only simple computational steps remaining.)
\end{problem}


\end{document}
%sagemathcloud={"zoom_width":100}