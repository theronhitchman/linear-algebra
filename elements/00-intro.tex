\documentclass[elementsmain.tex]{subfiles}
\begin{document}
\section*{Introduction: To The Student}

So. 
I am writing this book. 
(Let's not pretend I am done. 
You can see the state it is in.) 
There are dozens (hundreds?) of introductory linear algebra books, so it is pretty reasonable to ask why I am putting in the effort, and, in the meantime, causing this much pain. 
I should explain.

I have taught linear algebra many times and I have liked some books, but never loved one. 
The closest match for what I wanted to teach is Strang's \emph{Introduction to Linear Algebra}, and I am sure that people who have read that will see some influences here. 
But my students never seemed to connect with Prof.~Strang's enthusiastic, stream-of-consciousness prose. 
And over time, I found that the things I need to emphasize for my students just don't match with that text, or any other.

In addition, most textbooks assume a certain class structure: lectures accompanied by weekly homework, with some exams. 
I don't want to run our course that way. 

So this book is my solution. 
It is my attempt to make a thing which matches how I want our class to run. 

%Here is what you should expect. 
%This book has the basics of linear algebra, done thoroughly. 
%I want this to help you see why some very basic, important things are done the way they are. 
%It is amazing how much of the subject of linear algebra can be done by focusing on small examples, that is, in small dimensions. 
%(We'll learn about \emph{dimension} later on.) 
%We'll sort out other things through assignments and class discussions.

It is important to read this book actively. 
If you haven't learned how to read a math text before, there are some key ideas:
\begin{description}
\item[Time] Mathematics is often technical and tricky. 
It takes time to absorb. 
Plan to give yourself lots of time to read and think. 
And don't be surprised if you have to read some section more than once. 
(This is not a novel. 
As much as I see it as a story, it won't sweep you away.)
\item[Examples] In the interest of brevity, I have streamlined the exposition. 
In particular, there are no examples. 
\textbf{The point is that you should make your own.} 
This is so important a skill that it is basically a mathematical super-power. 
Whenever you come across an idea, if you understand it or not, you should make some very explicit examples and consider them carefully. 
\item[Questions] As part of your \emph{active} engagement with the text, you will find things that don't quite make sense, yet. 
This is normal. 
The mathematician's best approach then is to (1) write down a specific question or two about the confusing bit, and (2) talk to other people about it. 
You are fortunate that you have an instructor and classmates to talk to. 
Make lists of questions and try to get them answered! 
\end{description}

The real beauty in linear algebra is the tight set of connections between algebra and geometry.
I hope you enjoy it.


\clearpage
\end{document}